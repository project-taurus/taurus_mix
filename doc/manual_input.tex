%=======================================================================
% Manual concerning the format of the hamiltonian files.
%=======================================================================
%
%= Document class, packages and commands ===============================
%
\documentclass[a4paper,11pt]{article}
%
\usepackage[left=2.0cm,right=2.0cm,top=2.0cm,bottom=2.0cm,includefoot,includehead,headheight=13.6pt]{geometry}
\usepackage[T1]{fontenc}
\usepackage[utf8]{inputenc}
\usepackage{amsmath}
\usepackage[colorlinks,urlcolor=blue,linkcolor=blue]{hyperref}

\newcommand{\TAURUSpav}{$\text{TAURUS}_{\text{pav}}$}
\newcommand{\TAURUSmix}{$\text{TAURUS}_{\text{mix}}$}
\newcommand{\TAURUSmixt}{$\text{TAURUS}_{\text{mix}}$~}
\newcommand{\ttt}[1]{\texttt{#1}}

\newcommand{\bra}[1]{\langle #1 \vert}
\newcommand{\ket}[1]{\vert #1 \rangle}
\newcommand{\elma}[3]{\bra{#1} #2 \ket{#3}}

% 
% 
%= Begin document ======================================================
%
\begin{document}

%
%= Title =======================================
% 
\begin{center}
 {\LARGE \textbf{\TAURUSmix: Manual for the input files}} \\[0.20cm]
 {\large 14/05/2023}
\end{center}

%
%= Section: standard input =============================================
% 
\section{Structure of the standard input file}

The input file is read by the code as STDIN and therefore has no fixed naming convention. 
On the other hand, the format of the file is fixed. We describe in this manual how to write a proper input file for \TAURUSmix~
and the different options that the code offers.
We will present the different section of the input file separately (but remember that they
are all part of the same file).
 
\noindent Before going further, a few remark are in order:
\begin{itemize}
  \item We use here the Fortran convention for the format of variables, e.g.\ 1i5 or 1a30, assuming that the reader
  know their meaning. If it is not the case, we encourage the reader to search for a tutorial somewhere else.
  There is a large body of Fortran documentation available online therefore it should not be too difficult.

  \item All lines starts with the reading of an element of an array, \ttt{input\_names} or \ttt{input\_spec},
  made of \ttt{character(30)} variables. 
  These variables only play a cosmetic/descriptive role when reading or printing the input parameters and can be
  set to any desired value by the user. Therefore, we are not going to comment them further.

  \item The number of lines in the input files depends on the number of ``specific'' cut-offs used. Here, we will give
  the line numbers considering two specific cut-offs.
\end{itemize}

%
%= subsection: general parameters ======================================
%
\subsection{Section about the general parameters}

\subsubsection*{Description}
\begin{center}
\begin{tabular}{|c|l|l|}
\hline
Line & Format & Data \\
\hline
 \textbf{1}   & 1a             & \tt input\_names(1)                   \\
 \textbf{2}   & 1a             & \tt input\_names(2)                   \\
 \textbf{3}   & 1a30, 1i1      & \tt input\_names(3), hwg\_phys        \\
 \textbf{4}   & 1a30, 1i1      & \tt input\_names(4), hwg\_algo        \\
 \textbf{5}   & 1a30, 1i1      & \tt input\_names(5), hwg\_norm        \\
 \textbf{6}   & 1a30, 1i1      & \tt input\_names(6), hwg\_rmev        \\
 \textbf{7}   & 1a30, 1i1      & \tt input\_names(7), hwg\_conv        \\
 \textbf{8}   & 1a30, 1i3      & \tt input\_names(8), hwg\_Edis        \\
 \textbf{9}   & 1a             & \tt input\_names(9)                   \\
\hline
\end{tabular}
\end{center}
where 
\begin{itemize}
\item \ttt{hwg\_phys}: physical case studied. \\[0.05cm]
 \ttt{= 0\:} excitation spectrum and $\gamma$ spectrosocpy (ELM transiions: E0-3, M1-2). 
\item \ttt{hwg\_algo}: algorithm used to solve the HWG equation. \\[0.05cm]
 \ttt{= 0\:} reduction to a standard eigenvalue problem through the calculation of the square root of the norm matrix. \\[0.05cm]
 \ttt{= 1\:} QZ algorithm (as implemented in LAPACK).
\item \ttt{hwg\_norm}: option to normalize the projected matrix elements before solving the HWG equation. \\[0.05cm]
 \ttt{= 0\:} no normalization is performed. \\[0.05cm]
 \ttt{= 1\:} normalization using the projected overlaps.
\item \ttt{hwg\_rmev}: option to remove the projected states giving large negative eigenvalues, i.e.\ smaller than -\ttt{cutoff\_algo}, determined from incremental
  diagonalization of the norm. Only referenced for \ttt{hwg\_algo = 0}. \\[0.05cm]
 \ttt{= 0\:} no removal is performed. \\[0.05cm]
 \ttt{= 1\:} removes the projected states.
\item \ttt{hwg\_conv}: option to perform an anlysis of the convergence as a function of the number of norm eigenvalues considered. Only referenced for \ttt{hwg\_algo = 0}. \\[0.05cm]
 \ttt{= 0\:} no analysis is performed. \\[0.05cm]
 \ttt{= 1\:} For each $J^p$ block, the HWG equation is solved several times incrementally adding more norm eigenvalues. The results
\item \ttt{hwg\_Edis}: Maximum exictation energy displayed in the final tables containing the information on the spectrum and transitions.
 Note that separate files are created that contain the spectrum/transitions up to much higher excitation energy. 
\end{itemize}

\subsubsection*{Example}
\begin{center}
\tt
\begin{tabular}{|ll|}
\hline
General parameters            &     \\
------------------------------------ & \\
Physics case studied          &0    \\
Algorithm to solve HWG eq.    &0    \\
Normalization of matrices     &1    \\
Remove states giving ev<0     &1    \\
Convergence analysis (norm)   &0    \\
Max(E\_exc) displayed (Mev)   &10   \\
                              &     \\
\hline
\end{tabular}
\end{center}

%
%= subsection: quantum numbers =========================================
%
\subsection{Section about the quantum numbers}

\subsubsection*{Description}
\begin{center}
\begin{tabular}{|c|l|l|}
\hline
Line & Format & Data \\
\hline
 \textbf{10}   & 1a           & \tt input\_names(10)              \\
 \textbf{11}   & 1a           & \tt input\_names(11)              \\
 \textbf{12}   & 1a30, 1i3    & \tt input\_names(12), hwg\_Z      \\
 \textbf{13}   & 1a30, 1i3    & \tt input\_names(13), hwg\_N      \\
 \textbf{14}   & 1a30, 1i3    & \tt input\_names(14), hwg\_Zc     \\
 \textbf{15}   & 1a30, 1i3    & \tt input\_names(15), hwg\_Nc     \\
 \textbf{16}   & 1a30, 1i3    & \tt input\_names(16), hwg\_2jmax  \\
 \textbf{17}   & 1a30, 1i3    & \tt input\_names(17), hwg\_2jmin  \\
 \textbf{18}   & 1a30, 1i3    & \tt input\_names(18), hwg\_pmin   \\
 \textbf{19}   & 1a30, 1i3    & \tt input\_names(19), hwg\_pmax   \\
 \textbf{20}   & 1a30, 1f5.2  & \tt input\_names(20), hwg\_echp   \\
 \textbf{21}   & 1a30, 1f5.2  & \tt input\_names(21), hwg\_echn   \\
 \textbf{22}   & 1a           & \tt input\_names(22)              \\
\hline
\end{tabular}
\end{center}
where
\begin{itemize}
 \item \ttt{hwg\_Z}: number of active protons (after particle-number projection on the ket).
 \item \ttt{hwg\_N}: number of active neutrons (same as above). 
 \item \ttt{hwg\_Zc}: number of core protons.
 \item \ttt{hwg\_Nc}: number of core neutrons. 
 \item \ttt{hwg\_2jmin}: minimum value of the angular momentum $2J$ considered.
 \item \ttt{hwg\_2jmax}: maximum value of the angular momentum $2J$ considered.
 \item \ttt{hwg\_pmin}: minimum value of the parity $P$ considered.
 \item \ttt{hwg\_pmax}: maximum value of the parity $P$ considered.
 \item \ttt{hwg\_echp}: electric charge for the protons.              
 \item \ttt{hwg\_echn}: electric charge for the neutrons.             
\end{itemize}

\subsubsection*{Example}
\begin{center}
\tt
\begin{tabular}{|ll|}
\hline
Quantum numbers                &      \\
------------------------------ &      \\
Number of active protons  Z    &20    \\
Number of active neutrons N    &28    \\
Number of core protons  Zc     &0     \\
Number of core neutrons Nc     &0     \\
Angular momentum min(2*J)      &0     \\
Angular momentum max(2*J)      &8     \\
Parity min(P)                  &1     \\
Parity max(P)                  &1     \\
Electric charge protons\phantom{ } (*e)  &1.00  \\
Electric charge neutrons (*e)  &0.00  \\
                               &      \\
\hline
\end{tabular}
\end{center}

%
%= subsection: cutoffs =================================================
%
\subsection{Section about the cut-offs}

\subsubsection*{Description}
\begin{center}
\begin{tabular}{|c|l|l|}
\hline
Line & Format & Data \\
\hline
 \textbf{23}   & 1a    & \tt input\_names(23)                          \\
 \textbf{24}   & 1a    & \tt input\_names(24)                          \\
 \textbf{25}   & 1a30, 1i5 & \tt input\_names(25), cutoff\_ldim        \\
 \textbf{26}   & 1a30, 1es10.3 & \tt input\_names(26), cutoff\_over    \\
 \textbf{27}   & 1a30, 1es10.3 & \tt input\_names(27), cutoff\_algo    \\
 \textbf{28}   & 1a30, 1es10.3 & \tt input\_names(28), cutoff\_negev   \\
 \textbf{29}   & 1a30, 1es10.3 & \tt input\_names(29), cutoff\_J       \\
 \textbf{30}   & 1a30, 1es10.3 & \tt input\_names(30), cutoff\_A       \\
 \textbf{31}   & 1a30, 1i5 & \tt input\_names(31), cutoff\_spec\_dim   \\
               &           & \tt do i=1, cutoff\_spec\_dim           \\
 \textbf{  }   & 1a30, 1a1,  & \:\: \tt input\_spec(i), cutoff\_spec\_type(i),   \\
 \textbf{  }   & 2i3,        & \:\: \tt cutoff\_spec\_2j(i), cutoff\_spec\_p(i), \\
 \textbf{  }   & 1x, 1es10.3 & \:\: \tt cutoff\_spec\_value(i) \\
 \textbf{  }   &             & \:\: \tt (if cutoff\_spec\_type(i) $\ne$ 'L') \\
 \textbf{  }   &  OR         & \:\: OR \\
 \textbf{  }   & 1a30, 1a1,  & \:\: \tt input\_spec(i), cutoff\_spec\_type(i),   \\
 \textbf{  }   & 2i3,        & \:\: \tt cutoff\_spec\_2j(i), cutoff\_spec\_p(i), \\
 \textbf{  }   & 1i19,1i3    & \:\: \tt cutoff\_spec\_label(i), cutoff\_spec\_lab2k(i) \\
 \textbf{  }   &             & \:\: \tt (if cutoff\_spec\_type(i) $=$ 'L') \\
               &           & \tt enddo                               \\
\hline
\end{tabular}
\end{center}
where
\begin{itemize}
\item \ttt{cutoff\_ldim}: Maximum number of reference states considered in the mixing.
\item \ttt{cutoff\_over}: Default cut-off value for the projected overlap. The projected states with an overlap $<$ \ttt{cutoff\_over} will be discarded when reading the projected matrix elements.
\item \ttt{cutoff\_algo}: Default cut-off value for the norm eigenvalues. For \ttt{hwg\_algo = 0}, the norm eigenstates with an eigenvalue $<$ \ttt{cutoff\_algo} will be discarded when 
      building the natural basis. For \ttt{hwg\_algo = 1}, the generalized eigenvalues ($\alpha,\beta$), with $\beta <$ \ttt{cutoff\_algo} will be discarded.
\item \ttt{cutoff\_negev}: Default cut-off value for the negative eigenvalues. For \ttt{hwg\_rmev = 1}, the projected states giving a negative eigenvalue $<$ $-$\ttt{cutoff\_negev} 
      will be discarded before solving the HWG equation.
\item \ttt{cutoff\_J}: Default cut-off value for the expectation values of $J_z$ and $J^2$. The projected states with an expectation value $|\langle J_z/J^2 \rangle - K/J^2| >$ \ttt{cutoff\_J}
      will be discarded when reading the projected matrix elements.
\item \ttt{cutoff\_A}: Default cut-off value for the expectation values of $N$, $Z$ and $A$. The projected states with an expectation value $|\langle N/Z/A \rangle - N/Z/A| >$ \ttt{cutoff\_A}
      will be discarded when reading the projected matrix elements.
\item \ttt{cutoff\_spec\_dim}: Number of specific cut-offs, i.e.\ cut-offs that are specific to a given angular momentum/parity block.
\item \ttt{cutoff\_spec\_type}: Type of specific cut-off. \\[0.05cm]
 \ttt{= O\:} overlap cut-off. \\[0.05cm]
 \ttt{= S\:} small norm eigenstate cut-off. \\[0.05cm]
 \ttt{= N\:} negative eigenvalue cut-off. \\[0.05cm]
 \ttt{= J\:} $J_z$ and $J^2$ cut-off. \\[0.05cm]
 \ttt{= A\:} $N$, $Z$ and $A$ cut-off. \\[0.05cm]
 \ttt{= L\:} label cut-off. 
\item \ttt{cutoff\_spec\_2j}: 2*$J$ of the block considered.
\item \ttt{cutoff\_spec\_p}: $P$ of the block considered.
\item \ttt{cutoff\_spec\_value}: Value of the cut-off (if 'O', 'S', 'N', 'J' or 'A').
\item \ttt{cutoff\_spec\_label}: Label of the state (if 'L').
\item \ttt{cutoff\_spec\_lab2k}: 2*$K$ of the state (if 'L').
\end{itemize}

\subsubsection*{Example}
\begin{center}
\tt
\begin{tabular}{|ll|}
\hline
Cut-offs                      &          \\
----------------              &          \\
Maximum number of states      &0         \\
Cut-off projected overlap     &1.000E-06 \\
Cut-off norm eigenvalues      &1.000E-06 \\
Cut-off expect.~val.~Jz/J$^2$ &1.000E-02 \\
Cut-off expect.~val.~N/Z/A    &1.000E+01 \\
No.~of specialized cut offs   &2         \\
Example of label cut-off      &L \phantom{0}0 \phantom{0}1 1234567891234567891 \phantom{0}0  \\
Example of overlap cut-off    &O \phantom{0}0 \phantom{0}1 1.000E-05 \\
\hline
\end{tabular}
\end{center}

%
%= Section: other inputs ===============================================
% 
\section{Other input files}

The code \TAURUSmix~ takes in entry several binary input files produced by \TAURUSpav: 
\begin{itemize}
  \item \ttt{projmatelem\_states.bin:} contains the information about the projected states (e.g. expectation values for the Hamiltonian, quantum numbers and radii).
  \item \ttt{projmatelem\_Ql.bin (Ql=E1,E2,E3,M1,M2):} contains the informations about the electromagnetic transitions. These files are not necessary and the code will 
        skip the calculations of a given multipole if its associated file is missing. 
\end{itemize}
To gather all the files coming from different runs of \TAURUSpav, we recommend using the ``cat'' command. For example: \\
{\tt touch projmatelem\_states.bin} \\
{\tt cat state1\_projmatelem\_states.bin >> projmatelem\_states.bin} \\
{\tt cat state2\_projmatelem\_states.bin >> projmatelem\_states.bin} \\
{\tt $\ldots$} \\
The files can be gathered in any arbitrary order.

%
%= End document ======================================================
%
\end{document}
%
%=====================================================================
